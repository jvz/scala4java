\documentclass{beamer}

\mode<presentation>
{
  \usetheme{Warsaw}
}

\usepackage{listings}
\usepackage[english]{babel}
\usepackage[latin1]{inputenc}
\usepackage{times}
\usepackage[T1]{fontenc}

\title{Scala for Java Developers}
\subtitle{The Scalable Language for Everyone}
\author[Matt Sicker @jvz]{Matt Sicker \\ <\url{http://musigma.org/}>}
\institute{SPR Consulting}
\date{11 July 2017}

% If you have a file called "university-logo-filename.xxx", where xxx
% is a graphic format that can be processed by latex or pdflatex,
% resp., then you can add a logo as follows:

% \pgfdeclareimage[height=0.5cm]{university-logo}{university-logo-filename}
% \logo{\pgfuseimage{university-logo}}

% Delete this, if you do not want the table of contents to pop up at
% the beginning of each subsection:
\AtBeginSubsection[]
{
  \begin{frame}<beamer>{Outline}
    \tableofcontents[currentsection,currentsubsection]
  \end{frame}
}
% end

\begin{document}
\lstset{language=Scala}

\begin{frame}
  \titlepage
\end{frame}

\begin{frame}{Outline}
  \tableofcontents
\end{frame}

%%% BASIC SYNTAX
\section{Scala Syntax}

%% EXPRESSIONS
\subsection{Expressions}

\begin{frame}[fragile]{Variables}
\begin{itemize}
\item Variables are declared with \lstinline{val}
\item The type of the variable can oftentimes be inferred by the compiler
\item There are very little restrictions on allowed characters in identifiers
\item Use the \lstinline{_} variable as a throwaway variable
\end{itemize}
\begin{lstlisting}
val foo = "bar" // foo is inferred to be a String
\end{lstlisting}
\end{frame}

\begin{frame}{Underscores}
\begin{itemize}
\item The \lstinline{_} character is pervasive in Scala
\item In general, it means ``that thing, you know the one I'm talking about''
\item Used for throwaway variables, wildcard imports, anonymous lambda parameters,
eta expansion, possibly other Greek letters
\end{itemize}
\end{frame}

\begin{frame}{Semicolons}
\begin{itemize}
\item Where are all the semicolons?
\item Answer: they're optional! Don't bother using them outside of very specific use cases.
\item Scala infers semicolons in a much more sane fashion than JavaScript.
\end{itemize}
\end{frame}

\begin{frame}[fragile]{Types}
\begin{itemize}
\item Types come after the variable name
\item Variable types can oftentimes be inferred by the compiler
\item All types descend from \lstinline{Any} and are all supertypes of \lstinline{Nothing}
\item The \lstinline{AnyRef} class is an alias for \lstinline{java.lang.Object}
\item The \lstinline{AnyVal} class describes value-type classes such as the primitive types
\end{itemize}
\begin{lstlisting}
val name: String = "Matt"
\end{lstlisting}
\end{frame}

\begin{frame}[fragile]{Tuples}
\begin{itemize}
\item Tuples are generalized ordered pairs
\item Syntax sugar for instances of the \lstinline{TupleN} traits
\item
\begin{lstlisting}
val foo: (String, Int, Boolean) =
  ("Hello", 42, true)
\end{lstlisting}
\item Can access individual parts using the \lstinline{_N} methods or via destructuring:
\item
\begin{lstlisting}
val (s, _, _) = foo
val t = foo._2
// s = "Hello", t = 42
\end{lstlisting}
\end{itemize}
\end{frame}

\begin{frame}[fragile]{Methods}
\begin{itemize}
\item Methods are defined with \lstinline{def}
\item Can omit the return type if it's inferable
\item Must return something or \lstinline{Unit}
\item Can define methods inside methods
\item Can omit parenthesis for 0-arg methods
\item The last line(s) of reachable code in the method is the default return value
\end{itemize}
\begin{lstlisting}
def greet(name: String): Unit = {
  def greeting(name: String): String =
    s"Hello, $name" // string templates
  println(greeting(name))
}
\end{lstlisting}
\end{frame}

\begin{frame}[fragile]{Lambda Functions}
\begin{itemize}
\item Similar syntax to \lstinline{def}, though without a method name, and with an arrow
\item Very similar syntax to Java, but replace \lstinline{->} with \lstinline{=>}
\item Unlike Java, Scala lambdas can close over mutable variables
\item Syntactical sugar for an anonymous class of \lstinline{FunctionN} traits
\item In Scala 2.12, lambdas were updated to be compatible with Java 8 lambdas
\end{itemize}
\end{frame}

\begin{frame}[fragile]{Lambda Example}
\begin{lstlisting}
// all equivalent:
val add: (Int, Int) => Int = (a, b) => a + b
val add: (Int, Int) => Int = _ + _
val add = (_: Int) + (_: Int)
val add = new Function2[Int, Int, Int] {
  override def apply(a: Int, b: Int): Int =
    a + b
}
// calling a lambda
val sum = add(1, 2)
// equivalent to
val sum = add.apply(1, 2)
\end{lstlisting}
\end{frame}

\begin{frame}[fragile]{Conditionals}
\begin{itemize}
\item An \lstinline{if} expression returns the last value of the matching branch similar
to the ternary operator in Java
\item
\begin{lstlisting}
def describe(n: Int): String =
  if (n % 2 == 0) "even" else "odd"
\end{lstlisting}
\item Can still perform side effects and return \lstinline{Unit}
\end{itemize}
\end{frame}

\begin{frame}[fragile]{Loops}
\begin{itemize}
\item Use \lstinline{for} in a foreach loop or to transform collections using \lstinline{yield}
\item Can combine with ranges to get an indexed for loop (e.g., \lstinline{for (i <- 0 until 10)})
\end{itemize}
\begin{lstlisting}
val langs = List("Java", "Scala", "Clojure")
val lower = for (lang <- langs)
  yield lang.toLowerCase
// equivalent to:
val lower = langs.map(lang => lang.toLowerCase)
\end{lstlisting}
\end{frame}

\begin{frame}{Pattern Matching}
\begin{itemize}
\item An expression can be matched in many ways:
\item Type of expression
\item Value of expression
\item Types or values within the expression
\item Uses \lstinline{match} and \lstinline{case}
\item Additional predicates using \lstinline{if}
\item Similar to a switch statement in Java
\item Protip: a block made up of \lstinline{case} expressions is an anonymous match and can be
used as a single-argument lambda function
\end{itemize}
\end{frame}

\begin{frame}[fragile]{Pattern Matching Example}
\begin{lstlisting}
def describe(x: Any): String =
  x match {
    case null => "null"
    case i: Int => i.toString
    case s: String if s.nonEmpty => s
    case Some(y) => describe(y)
    case None => "none"
    case _ => "unknown" // default case
  }
\end{lstlisting}
\end{frame}

\begin{frame}[fragile]{Exceptions}
\begin{itemize}
\item All exceptions are unchecked in Scala
\item Throw an exception with \lstinline{throw}
\item Catch exceptions using \lstinline{try} and \lstinline{catch}
\item A \lstinline{catch} block is a pattern match expression on the exception
\end{itemize}
\begin{lstlisting}
def open(file: String) =
  throw new Exception("File not found")
try {
  val f = open("foo.txt")
} catch {
  case e: Exception =>
    println(e.getMessage)
}
\end{lstlisting}
\end{frame}


%% CLASSES
\subsection{Classes}

\begin{frame}{Traits}
\begin{itemize}
\item A \lstinline{trait} is similar to an interface in Java
\item Defines abstract methods and fields
\item Can also define concrete methods and fields
\item Very similar to an abstract class but cannot have a constructor
\item Classes can inherit from multiple traits, but only from one class
\item Traits can be restricted to certain implementing types
\end{itemize}
\end{frame}

\begin{frame}[fragile]{Trait Example}
\begin{lstlisting}
trait Logger {
  def isEnabled(level: String): Boolean
  def append(level: String, message: Any): Unit
  def log(level: String, message: Any): Unit =
    if (isEnabled(level)) append(level, message)
  def error(message: Any): Unit =
    log("ERROR", message)
  def debug(message: Any): Unit =
    log("DEBUG", message)
}
\end{lstlisting}
\end{frame}

\begin{frame}{Classes}
\begin{itemize}
\item A \lstinline{class} works rather similarly to Java classes
\item Contains a main constructor and optionally other constructors named \lstinline{this}
\item The body of the class (minus any \lstinline{def}s) is the constructor
\item Fields can be exposed using \lstinline{val} and \lstinline{var} for read-only
and read-write properties
\item Can extend another class and several traits
\item Use \lstinline{extends} to extend a class or implement a trait
\item Use \lstinline{with} to add additional traits to mix in to the class
\item Use \lstinline{override} on methods and variables overridden from parent
\item Use \lstinline{lazy val} for values that aren't evaluated until first access
\end{itemize}
\end{frame}

\begin{frame}[fragile]{Class Example}
\begin{lstlisting}
class StdoutLogger(levels: Map[String, Boolean])
    extends Logger {
  override def isEnabled(level: String): Boolean =
    levels(level)
  // marking as final prevents subclasses
  // from overriding
  final override def append(
      level: String, message: Any): Unit =
    println(s"$level: $message")
}
val logger = new StdoutLogger(
  Map("ERROR" -> true, "DEBUG" -> false))
\end{lstlisting}
\end{frame}

\begin{frame}{Objects}
\begin{itemize}
\item Scala does not have a static keyword
\item It does however have a singleton \lstinline{object} keyword
\item An object is a class with only a single instance
\item When named the same as a class, provides similar semantics to having static
methods defined on the class itself
\end{itemize}
\end{frame}

\begin{frame}[fragile]{Object Example}
\begin{lstlisting}
object Logger {
  def apply(debug: Boolean, error: Boolean): Logger =
    new StdoutLogger(Map(
      "DEBUG" -> debug,
      "ERROR" -> error
    ))
}
val logger = Logger(false, true)
logger.debug("Test")
\end{lstlisting}
\end{frame}

\begin{frame}{Case Classes}
\begin{itemize}
\item In Java, there is a lot of boilerplate to create a simple data class
\item Using Lombok, we can avoid most of it by adding annotations like
\lstinline{@Data}, \lstinline{@Wither}, \lstinline{@Builder}, etc., to the class
\item In Scala, we can add \lstinline{case} to a \lstinline{class} to get
similar functionality generated for us: toString, equals, hashCode, copy,
apply, unapply, Scala-style getters for the constructor parameters, an all-args
constructor, and some other goodies
\end{itemize}
\end{frame}

\begin{frame}[fragile]{Case Class Example}
\begin{lstlisting}
case class Name(first: String, last: String)
// automatic Name.apply created
val john = Name("John", "Doe")
// automatic Name.unapply created
val Name(first, last) = john
def isAnon(n: Name): Boolean = n match {
  case Name(_, "Doe") => true
  case _ => false
}
// automatic Name.copy like @Wither
val jane = john.copy(first = "Jane")
\end{lstlisting}
\end{frame}

\begin{frame}{Generics}
\begin{itemize}
\item Unlike Java, Scala does not allow raw types
\item Generic syntax uses square brackets instead of angled brackets
\item For consistency, arrays use the same syntax as collections
\item \lstinline{val xs: Array[Int] = Array(1, 2, 3)}
\item Can specify how instances relate using generic type parameter by
specifying the variance (similar to super/extends in Java)
\item Can use type parameter bounds using \lstinline{>:} and \lstinline{<:}
for superclass and subclass respectively
\end{itemize}
\end{frame}

\begin{frame}[fragile]{Generic Example}
\begin{lstlisting}
// +A: if B extends A, then Bag[B] extends Bag[A]
trait Bag[+A] {
  // if B super A, then we widen the type
  def add[B >: A](b: B): Bag[B]
  def remove(p: A => Boolean): Bag[A]
  // defining map, flatMap, and foreach allow this
  // class to be used in various for expressions
  def map[B](f: A => B): Bag[B]
  def flatMap[B](f: A => Bag[B]): Bag[B]
  def foreach(f: A => Unit): Unit
}
\end{lstlisting}
\end{frame}

\begin{frame}[fragile]{Class and Method Parameters}
\begin{itemize}
\item The arguments to the primary constructor of a class can be considered the
class's arguments similar to a method
\item Arguments can be passed by name out of order, negating the need for builders:
\begin{lstlisting}
val log = Logger(error = true, debug = false)
\end{lstlisting}
\item Parameters can have a default value:
\begin{lstlisting}
def greet(name: String = "World") =
  s"Hello, $name"
val greeting1 = greet()
val greeting2 = greet("Chicago")
\end{lstlisting}
\end{itemize}
\end{frame}

\begin{frame}[fragile]{Repeated Parameters}
\begin{itemize}
\item Similar to varargs in Java, the last argument in a parameter list can be a
repeated parameter
\item Access the variable as a \lstinline{Seq[T]} collection class
\item Expand a collection class to a repeated parameter using \lstinline{: _*} on the variable
\end{itemize}
\begin{lstlisting}
def join(fields: String*): String =
  fields.mkString(",")
val csv1 = join("foo", "bar", "baz")
val csv2 = join(List("foo", "bar"): _*)
\end{lstlisting}
\end{frame}

\begin{frame}[fragile]{Multiple Parameter Lists}
\begin{itemize}
\item A class or method can contain multiple parameter lists
\item This syntax can be useful for partial function application
\end{itemize}
\begin{lstlisting}
trait Foldable[A] {
  def fold[B](init: B)(op: (B, A) => B)
}
val f: Foldable[Int] = ...
f.fold(0)((sum, next) => sum + next)
\end{lstlisting}
\end{frame}


%% IMPLICITS
\subsection{Implicits}

\begin{frame}[fragile]{Implicit Conversions}
\begin{itemize}
\item In many projects, common boilerplate code to convert from one type to another
\item In Scala, we can create an implicit conversion to automatically convert types
where applicable
\item Implicits are a core feature of Scala that differentiate it from other languages
\end{itemize}
\begin{lstlisting}
def foo(id: UUID): Unit = ...
implicit def s2id(s: String): UUID =
  UUID.fromString(s)
val id = "cea3e50d-f894-47fe-a31b-0cb57c94bea5"
foo(id)
// expands to:
foo(s2id(id))
\end{lstlisting}
\end{frame}

\begin{frame}[fragile]{Implicit Classes}
\begin{itemize}
\item In order to add methods to third party classes, we can wrap the class and
provide new methods
\item Combined with an implicit conversion, we can use a shorthand syntax to make
an implicit class
\item An implicit class is a class with a single parameter with a generated implicit
function to convert from the type of the parameter to the implicit class
\end{itemize}
\begin{lstlisting}
implicit class IntOps(i: Int) {
  def isEven: Boolean = i % 2 == 0
}
val q = 42.isEven
\end{lstlisting}
\end{frame}

\begin{frame}[fragile]{DSL Example}
\begin{lstlisting}
case class Module(group: String, module: String)
implicit class Group(group: String) {
  def %(artifact: String): Module =
    Module(group, artifact)
}
val module = "org.apache.commons" % "commons-lang3"
\end{lstlisting}
\end{frame}

\begin{frame}[fragile]{Implicit Parameters}
\begin{itemize}
\item Passing the same contextual information over and over again is repetitive
\item Using implicit parameters along with implicit values helps reduce boilerplate
\end{itemize}
\begin{lstlisting}
def fetch(query: String)
  (implicit conn: Connection): Seq[Row] =
  conn.query(query)
implicit val c: Connection = ...
val rows = fetch("select * from things")
// expands to:
val rows = fetch("select * from things")(c)
\end{lstlisting}
\end{frame}

\begin{frame}[fragile]{Implicit Context Bounds}
\begin{itemize}
\item Some types are used to provide context for another type such as
\lstinline{Ordering[T]} for defining an ordering on a type
\item We can provide context objects via implicit parameters
\end{itemize}
\begin{lstlisting}
def min[A](a: A, b: A)
  (implicit o: Ordering[A]): A =
  o.min(a, b)
// or we can add a context bound
// and summon the implicit:
def min[A: Ordering](a: A, b: A): A =
  implicitly[Ordering[A]].min(a, b)
\end{lstlisting}
\end{frame}

%%% SUMMARY
\section*{Summary}

\begin{frame}{Summary}
\begin{itemize}
\item Scala provides a small, consistent core language with lots of optional
syntax sugar
\item Works well with existing Java libraries
\item Eliminates a lot of common boilerplate
\item Java is slowly adopting old Scala features (lambda functions, streams, and
eventually pattern matching), so why wait?
\item This is only the basics; Scala provides a standard library with very rich
collection classes, more syntax sugar for functional programming, and many
production-ready libraries and frameworks
\end{itemize}
\end{frame}

\begin{frame}{Further Reading}
\begin{itemize}
\item \url{http://musigma.org/scala/2017/07/03/akka-cqrs.html}
\item \url{https://github.com/jvz/akka-blog-example}
\item \url{https://github.com/jvz/scala-for-java}
\end{itemize}
\end{frame}

\end{document}